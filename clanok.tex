\documentclass[12pt, letterpaper]{article}
\usepackage[utf8]{inputenc}

% Header & Footer 
\usepackage{fancyhdr}
\pagestyle{myheadings }



%

\begin{document}

\begin{titlepage}
	\begin{center}	
	\line(1,0){300} \\
	[0.25in]
	\huge{\bfseries \huge Spring boot ako ľahký nastroj pre vašu stránku} \\
	[2mm]
	\line(1,0){200} \\
	[0.5cm]
	\textsc{\LARGE Fakulta informatiky a informačných technológií STU} \\
	[7cm]
	\end{center}
	\begin{flushright}
	\textsc{\large Andrei Trusau \\
	ID: 116317 \\
	\ 02.11.2021 \\}
	\end{flushright}
\end{titlepage}

\section{Abstrakt}\label{sec:intro}[15cm]
fghjkjhgfdfghjkjhgfdfghjkjh
Spring je veľmi populárne vývojové prostredie založené na jazyku Java, ktoré sa používa na vytváranie webových a podnikových aplikácií. Na rozdiel od mnohých iných platforiem, ktoré sa zameriavajú iba na jednu oblasť, Spring poskytuje širokú škálu funkcií, ktoré spĺňajú potreby dnešného podnikania prostredníctvom svojich portfóliových projektov.

Spring framework poskytuje flexibilitu na prispôsobenie bean-komponentov rôznymi spôsobmi, ako je XML, anotácie a JavaConfig. S rastúcim počtom funkcií sa zvyšuje zložitosť a konfigurácia aplikácií Spring sa stáva únavnou a náchylnou na chyby.

Tím Spring vytvoril Spring Boot na riešenie zložitosti konfigurácie. \\

Ale predtým, než sa ponoríme do Spring Boot, rýchlo sa pozrieme na Spring Boot, aby sme zistili, aké problémy sa Spring Boot snaží vyriešiť.\\
[15cm]
\section{Úvod}
Ak ste vývojár v jazyku Java, väčšina z vás už o Spring počula a možno ste ju dokonca použili vo svojich projektoch. Spring väčšinou pochádza z kontajnera na vstrekovanie závislostí, ale je toho oveľa viac.







\title{Spring boot ako ľahký nastroj pre vašu stránku}
\maketitle

\section{Predstavujeme Spring Boot}

\section{Vlastnosti Spring Boot}

\section{Webové aplikácie na Spring Boot}

\section{Požiadavky na inštaláciu Spring Boot}

\section{Aký je rozdiel medzi Spring Boot a Spring Boot?}

\section{2}

\end{document}